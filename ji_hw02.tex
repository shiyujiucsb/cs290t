\documentclass[12pt]{article}
\usepackage{graphicx}
\usepackage{fullpage}
\usepackage{amsmath}
\usepackage{amsfonts}
\usepackage{amssymb}
\usepackage{amsthm}
\usepackage{algpseudocode}
\usepackage{algorithm}

\setlength{\parindent}{0cm} 

\newtheorem{proposition}{Proposition}
\newtheorem{claim}{Claim}
\theoremstyle{plain}

\begin{document}

\title{CS 290T Computational Algebra Homework Assignment 02}
\author{Shiyu Ji}
\date{\today}
\maketitle

\newcommand{\m}[1]{\begin{bmatrix}#1\end{bmatrix}}
\newcommand{\rank}[1]{\operatorname{rank}(#1)}
\newcommand{\F}{\mathbb{F}}

\section{Problem 1}
Let $T_f =  (0, 1, 0, 1, 0, 1, 1, 1, 1, 1, 0, 1, 0, 1, 1, 1)$ be the truth table of $f : \F_2^4 \to \F_2$.

1. Find the ANF (Algebraic Normal Form) of $f$.

{\bf Solution}: We can make a table:
\begin{quote}
\begin{tabular}{c c c c| c || r || r || r || r}
$x_1$ & $x_2$ & $x_3$ & $x_4$ & $f(x_1, x_2, x_3, x_4)$ & 1. & 2. & 3. & 4. \\
\hline
0 & 0 & 0 & 0 & 0 & 0 		&		0		& 0 		& 0\\
0 & 0 & 0 & 1 & 1 & 0+1=1 &		1		& 1 		& 1\\
0 & 0 & 1 & 0 & 0 & 0 		& 0+0=0	& 0 		& 0\\
0 & 0 & 1 & 1 & 1 & 0+1=1	&	1+1=0	& 0 		& 0\\
0 & 1 & 0 & 0 & 0 & 0			&		0		& 0+0=0 & 0\\
0 & 1 & 0 & 1 & 1 & 0+1=1	&		1		& 1+1=0 & 0\\
0 & 1 & 1 & 0 & 1 & 1			& 0+1=1	& 0+1=1 & 1\\
0 & 1 & 1 & 1 & 1 & 1+1=0	&	1+0=1	& 0+1=1 & 1\\
1 & 0 & 0 & 0 & 1 & 1			&		1		& 1 		& 0+1=1\\
1 & 0 & 0 & 1 & 1 & 1+1=0	&		0		& 0 		& 1+0=1\\
1 & 0 & 1 & 0 & 0 & 0			&	1+0=1 & 1 		& 0+1=1\\
1 & 0 & 1 & 1 & 1 & 0+1=1	&	0+1=1	& 1 		& 0+1=1\\
1 & 1 & 0 & 0 & 0 & 0			&		0		& 1+0=1 & 0+1=1\\
1 & 1 & 0 & 1 & 1 & 0+1=1	&		1		& 0+1=1 & 0+1=1\\
1 & 1 & 1 & 0 & 1 & 1			&	0+1=1	& 1+1=0 & 1+0=1\\
1 & 1 & 1 & 1 & 1 & 1+1=0	&	1+0=1	& 1+1=0 & 1+0=1\\
\end{tabular}
\end{quote}
Hence the ANF is: 
$$\begin{aligned}
f =& (1+x_1)(1+x_2)(1+x_3)x_4 + (1+x_1)(1+x_2)x_3 x_4 + (1+x_1)x_2(1+x_3)x_4 \\
&+ (1+x_1)x_2 x_3(1+x_4) + (1+x_1)x_2 x_3 x_4 + x_1(1+x_2)(1+x_3)(1+x_4) + x_1(1+x_2)(1+x_3)x_4 \\
&+ x_1(1+x_2)x_3 x_4 + x_1 x_2 (1+x_3) x_4 + x_1 x_2 x_3 (1+x_4) + x_1 x_2 x_3 x_4 \\
=& x_1x_2x_3x_4 + x_1x_2x_3 + x_1x_2x_4 + x_1x_3x_4 + x_2x_3x_4 + x_1x_2 
+ x_1x_3 + x_1x_4 + x_2x_3 + x_1 + x_4.
\end{aligned}$$

2. Find the degree of $f$.

{\bf Solution}: The degree is 4.

3. What is the distance between $f$ and $x_1 + x_3 + x_4$?

{\bf Solution}: The distance is given by the weight of the function
$$\begin{aligned}
&h := f + x_1 + x_3 + x_4 \\
=& x_1x_2x_3x_4 + x_1x_2x_3 + x_1x_2x_4 + x_1x_3x_4 + x_2x_3x_4 + x_1x_2 
+ x_1x_3 + x_1x_4 + x_2x_3 + x_2.
\end{aligned}$$
which is 7 given as follows.

\begin{quote}
\begin{tabular}{c c c c| c || c || c }
$x_1$ & $x_2$ & $x_3$ & $x_4$ & $f(x_1, x_2, x_3, x_4)$ & $x_1+x_3+x_4$ & $h(x_1, x_2, x_3, x_4)$ \\
\hline
0 & 0 & 0 & 0 & 0 & 0 &	0\\
0 & 0 & 0 & 1 & 1 & 1 &	0\\
0 & 0 & 1 & 0 & 0 & 1 & 1\\
0 & 0 & 1 & 1 & 1 & 0	&	1\\
0 & 1 & 0 & 0 & 0 & 0	&	0\\
0 & 1 & 0 & 1 & 1 & 1	&	0\\
0 & 1 & 1 & 0 & 1 & 1	& 0\\
0 & 1 & 1 & 1 & 1 & 0	&	1\\
1 & 0 & 0 & 0 & 1 & 1	&	0\\
1 & 0 & 0 & 1 & 1 & 0	&	1\\
1 & 0 & 1 & 0 & 0 & 0	&	0\\
1 & 0 & 1 & 1 & 1 & 1	&	0\\
1 & 1 & 0 & 0 & 0 & 1	&	1\\
1 & 1 & 0 & 1 & 1 & 0	&	1\\
1 & 1 & 1 & 0 & 1 & 0	&	1\\
1 & 1 & 1 & 1 & 1 & 1	&	0\\
\end{tabular}
\end{quote}

4. Find the nonlinearity of $f$.

{\bf Solution}: That corresponds to $f + x_4$, which is given by the following table. 

After trying all the 15 4-variable affine functions, $3$ is the least Hamming distance, which is the nonlinearity of $f$.

\begin{quote}
\begin{tabular}{c c c c| c || c || c }
$x_1$ & $x_2$ & $x_3$ & $x_4$ & $f(x_1, x_2, x_3, x_4)$ & $x_4$ & $f(x_1, x_2, x_3, x_4)+x_1+x_4$ \\
\hline
0 & 0 & 0 & 0 & 0 & 0 &	0\\
0 & 0 & 0 & 1 & 1 & 1 &	0\\
0 & 0 & 1 & 0 & 0 & 0 & 0\\
0 & 0 & 1 & 1 & 1 & 1	&	0\\
0 & 1 & 0 & 0 & 0 & 0	&	0\\
0 & 1 & 0 & 1 & 1 & 1	&	0\\
0 & 1 & 1 & 0 & 1 & 0	& 1\\
0 & 1 & 1 & 1 & 1 & 1	&	0\\
1 & 0 & 0 & 0 & 1 & 0	&	1\\
1 & 0 & 0 & 1 & 1 & 1	&	0\\
1 & 0 & 1 & 0 & 0 & 0	&	0\\
1 & 0 & 1 & 1 & 1 & 1	&	0\\
1 & 1 & 0 & 0 & 0 & 0	&	0\\
1 & 1 & 0 & 1 & 1 & 1	&	0\\
1 & 1 & 1 & 0 & 1 & 0	&	1\\
1 & 1 & 1 & 1 & 1 & 1	&	0\\
\end{tabular}
\end{quote}

5. Does $f$ satisfy $PC(1)$?

{\bf Solution}: Consider the function $f(\mathbf{x})+f(\mathbf{x}+x_1)$, where $\mathbf{x} = (x_1, x_2, x_3, x_4)$.

\begin{quote}
\begin{tabular}{c c c c| c || c || c }
$x_1$ & $x_2$ & $x_3$ & $x_4$ & $f(\mathbf{x})$ & $f(\mathbf{x}+x_1)$ & $f(\mathbf{x})+f(\mathbf{x}+x_1)$ \\
\hline
0 & 0 & 0 & 0 & 0 & 1 &	1\\
0 & 0 & 0 & 1 & 1 & 1 &	0\\
0 & 0 & 1 & 0 & 0 & 0 & 0\\
0 & 0 & 1 & 1 & 1 & 1	&	0\\
0 & 1 & 0 & 0 & 0 & 0	&	0\\
0 & 1 & 0 & 1 & 1 & 1	&	0\\
0 & 1 & 1 & 0 & 1 & 1	& 0\\
0 & 1 & 1 & 1 & 1 & 1	&	0\\
1 & 0 & 0 & 0 & 1 & 0	&	1\\
1 & 0 & 0 & 1 & 1 & 1	&	0\\
1 & 0 & 1 & 0 & 0 & 0	&	0\\
1 & 0 & 1 & 1 & 1 & 1	&	0\\
1 & 1 & 0 & 0 & 0 & 0	&	0\\
1 & 1 & 0 & 1 & 1 & 1	&	0\\
1 & 1 & 1 & 0 & 1 & 1	&	0\\
1 & 1 & 1 & 1 & 1 & 1	&	0\\
\end{tabular}
\end{quote}

The output has only 2 1's but 14 0's. Thus it is not balanced. Hence $f$ does not satisfy $PC(1)$.

\section{Problem 2}
Let $f(x_1, x_2, x_3, x_4) = x_1 + x_2x_3 + x_1x_4$ be a Boolean function over $\F_2^4$.

1. Find the truth table of $f$.

\begin{quote}
\begin{tabular}{c c c c| c }
$x_1$ & $x_2$ & $x_3$ & $x_4$ & $f(x_1, x_2, x_3, x_4)$ \\
\hline
0 & 0 & 0 & 0 & 0 \\
0 & 0 & 0 & 1 & 0 \\
0 & 0 & 1 & 0 & 0 \\
0 & 0 & 1 & 1 & 0 \\
0 & 1 & 0 & 0 & 0 \\
0 & 1 & 0 & 1 & 0 \\
0 & 1 & 1 & 0 & 1 \\
0 & 1 & 1 & 1 & 1 \\
1 & 0 & 0 & 0 & 1 \\
1 & 0 & 0 & 1 & 0 \\
1 & 0 & 1 & 0 & 1 \\
1 & 0 & 1 & 1 & 0 \\
1 & 1 & 0 & 0 & 1 \\
1 & 1 & 0 & 1 & 0 \\
1 & 1 & 1 & 0 & 0 \\
1 & 1 & 1 & 1 & 1 \\
\end{tabular}
\end{quote}

2. What is the distance between $f$ and $x_1 + x_2$?

{\bf Solution}: The distance is the weight of the function
$$h := f+x_1 + x_2 = x_2 + x_2x_3 + x_1x_4.$$

\begin{quote}
\begin{tabular}{c c c c| c }
$x_1$ & $x_2$ & $x_3$ & $x_4$ & $h(x_1, x_2, x_3, x_4)$ \\
\hline
0 & 0 & 0 & 0 & 0 \\
0 & 0 & 0 & 1 & 0 \\
0 & 0 & 1 & 0 & 0 \\
0 & 0 & 1 & 1 & 0 \\
0 & 1 & 0 & 0 & 1 \\
0 & 1 & 0 & 1 & 1 \\
0 & 1 & 1 & 0 & 0 \\
0 & 1 & 1 & 1 & 0 \\
1 & 0 & 0 & 0 & 0 \\
1 & 0 & 0 & 1 & 1 \\
1 & 0 & 1 & 0 & 0 \\
1 & 0 & 1 & 1 & 1 \\
1 & 1 & 0 & 0 & 1 \\
1 & 1 & 0 & 1 & 0 \\
1 & 1 & 1 & 0 & 0 \\
1 & 1 & 1 & 1 & 1 \\
\end{tabular}
\end{quote}
Thus the distance is 6.

3. Find the nonlinearity of $f$.

{\bf Solution}: We have tried every 4-variable affine function, and find that the minimum distance is 6, e.g., with $x_1 + x_2$. Thus the nonlinearity of $f$ is 6.

4. Does $f$ satisfy $SAC$?

{\bf Solution}: $SAC$ is $PC(1)$. We may check all the 4 one-weight affine functions: $x_1$, $x_2$, $x_3$ and $x_4$. The weight of each $f+x_i$ is always 6, which is $2^{n-1} - 2^{n/2-1}$, $n=4$. Thus $f$ is bent and every bent function must satisfy $SAC$.


\section{Problem 3}
Let $f$ be any Boolean function from $\F_2^8$ to $\F_2$ and let 
$$W_f(a) = \sum_{x\in \F_2^8} (-1)^{f(x)+a\cdot x}$$
denote its Walsh transform $(a\cdot x = \sum_{i=1}^8 a_i x_i)$. Then find the maximum and minimum value of
$$\sum_{a \in \F_2^8}(W_f(a))^2.$$

{\bf Solution}: By setting $f(x)=0$ for all $x$, a simple computation gives 
$$\sum_{a \in \F_2^8}(W_f(a))^2 = (2^8)^2 = 2^{16}.$$
In fact, this relation holds for all the functions $f$. Thus the maximum and minimum values are the same: $2^{16}$. 
Next we prove a stronger claim, which is also known as Parseval's relation.

\begin{claim}
For all the functions $f : \F_2^n \to \F_2$,
$$\sum_{a \in \F_2^n}(W_f(a))^2 = 2^{2n},$$
where
$$W_f(a) := \sum_{x\in \F_2^n} (-1)^{f(x)+a\cdot x}.$$
\end{claim}
\begin{proof}
$$\begin{aligned}
& \sum_{a \in \F_2^n}(W_f(a))^2 = \sum_{a \in \F_2^n} \left( \sum_{x\in \F_2^n} (-1)^{f(x)+a\cdot x} \right)^2 \\
=& \sum_{a \in \F_2^n} \sum_{x\in \F_2^n} \left( (-1)^{f(x)+a\cdot x} \right)^2 + \sum_{a \in \F_2^n} \sum_{x\not=x'} (-1)^{f(x)+a\cdot x}(-1)^{f(x')+a\cdot x'} \\
=& 2^{2n} + \sum_{a \in \F_2^n} \sum_{x\not=x'} (-1)^{f(x)+a\cdot x}(-1)^{f(x')+a\cdot x'}. \\
\end{aligned}$$
It suffices to show
\begin{equation}
\label{eq:0sum}
\sum_{a \in \F_2^n} \sum_{x\not=x'} (-1)^{f(x)+a\cdot x + f(x')+a\cdot x'} = 0.
\end{equation}
Since $x\not=x'$, $x+x'\not= 0$, and thus there exists one bit of $x+x'$ that is not 0. W.l.o.g assume $i$-th bit of $x+x'$ is 1, and thus $e_i \cdot (x + x') = 1$. Let $a' = a + e_i$. 
Then
$$a'\cdot(x+x') = (a+e_i)\cdot(x+x') = a\cdot(x+x') + e_i\cdot(x+x') = a\cdot(x+x') + 1.$$
Thus
$$(-1)^{a\cdot(x+x')} + (-1)^{a'\cdot(x+x')} = 0.$$
Note that $a = a' + e_i$. Thus there always exists such a pair $(a,a')$ for each $(x,x')$. The sum cancels out over each pair. Hence the zero total sum in (\ref{eq:0sum}) follows.
\end{proof}
%{\bf Solution}: Note that
%$$0 \leq N(f) = 2^{n-1} -\frac{1}{2}\max_{a\in\F_2^n}|W_f(a)| \leq 2^{n-1} - 2^{n/2-1},$$
%where $n=8$.
%Hence
%$$2^{n/2} \leq \max_{a\in\F_2^n}|W_f(a)| \leq 2^{n}.$$
%Note that
%$$\sum_{a \in \F_2^8}(W_f(a))^2 \geq (\max_{a\in\F_2^8}|W_f(a)|)^2 \geq 2^8 = 256.$$
%This minimum value can be achieved by setting $f(x)=0$ for all $x$.
%
%On the other hand, 
%$$\sum_{a \in \F_2^8}(W_f(a))^2 \leq 2^8 \cdot (\max_{a\in\F_2^8}|W_f(a)|)^2 \leq 2^{24}.$$
%Since $a\cdot x$ is an affine function, a bent function $f$ can maximize each $|W_f(a)|$ to be $2^{n-1} - 2^{n/2-1}$, $n=8$. Hence
%$$\sum_{a \in \F_2^8}(W_f(a))^2 \leq 2^8 \cdot (2^7-2^3)^2 = 3686400.$$

\section{Problem 4}
The \emph{linear complexity} of a given binary sequence is the length of the
smallest LFSR (one with the fewest states) that produces the sequence.
Compute the linear complexity of the following sequence:
$$000100100011010001010110011110001001$$

{\bf Solution}: Use Berlekamp-Massey algorithm.

We find that the 18-th determinant is the last non-zero one. 

Hence we need to calculate
$$
\left[
\begin{array}{*{18}c}
0&0&0&1&0&0&1&0&0&0&1&1&0&1&0&0&0&1 \\
0&0&1&0&0&1&0&0&0&1&1&0&1&0&0&0&1&0 \\
0&1&0&0&1&0&0&0&1&1&0&1&0&0&0&1&0&1 \\
1&0&0&1&0&0&0&1&1&0&1&0&0&0&1&0&1&0 \\
0&0&1&0&0&0&1&1&0&1&0&0&0&1&0&1&0&1 \\
0&1&0&0&0&1&1&0&1&0&0&0&1&0&1&0&1&1 \\
1&0&0&0&1&1&0&1&0&0&0&1&0&1&0&1&1&0 \\
0&0&0&1&1&0&1&0&0&0&1&0&1&0&1&1&0&0 \\
0&0&1&1&0&1&0&0&0&1&0&1&0&1&1&0&0&1 \\
0&1&1&0&1&0&0&0&1&0&1&0&1&1&0&0&1&1 \\
1&1&0&1&0&0&0&1&0&1&0&1&1&0&0&1&1&1 \\
1&0&1&0&0&0&1&0&1&0&1&1&0&0&1&1&1&1 \\
0&1&0&0&0&1&0&1&0&1&1&0&0&1&1&1&1&0 \\
1&0&0&0&1&0&1&0&1&1&0&0&1&1&1&1&0&0 \\
0&0&0&1&0&1&0&1&1&0&0&1&1&1&1&0&0&0 \\
0&0&1&0&1&0&1&1&0&0&1&1&1&1&0&0&0&1 \\
0&1&0&1&0&1&1&0&0&1&1&1&1&0&0&0&1&0 \\
1&0&1&0&1&1&0&0&1&1&1&1&0&0&0&1&0&0 \\
\end{array}
\right]
\mathbf{x} = \m{
0\\1\\0\\1\\1\\0\\0\\1\\1\\1\\1\\0\\0\\0\\1\\0\\0\\1\\
}.
$$
Solve $\mathbf{x}$:
$$\mathbf{x} = (
  		1,
       0,
       1,
       1,
       0,
       0,
       0,
       1,
       1,
       1,
       0,
       1,
       1,
       1,
       1,
       0,
       0,
       1
)^T.$$
Thus the smallest LFSR is:
$$P(x) = x^{18} + x^{17} + x^{14} + x^{13} + x^{12} + x^{11} + x^9 + x^8 + x^7 + x^3 + x^2 + 1.$$
Hence the linear complexity is 18.

\end{document}
