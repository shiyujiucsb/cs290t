\documentclass[12pt]{article}
\usepackage{graphicx}
\usepackage{fullpage}
\usepackage{amsmath}
\usepackage{amsfonts}
\usepackage{amssymb}
\usepackage{amsthm}
\usepackage{algpseudocode}
\usepackage{algorithm}

\setlength{\parindent}{0cm} 

\newtheorem{proposition}{Proposition}
\newtheorem{claim}{Claim}
\theoremstyle{plain}

\begin{document}

\title{CS 290T Computational Algebra Homework Assignment 03}
\author{Shiyu Ji}
\date{\today}
\maketitle

\newcommand{\m}[1]{\begin{bmatrix}#1\end{bmatrix}}
\newcommand{\rank}[1]{\operatorname{rank}(#1)}
\newcommand{\F}{\mathbb{F}}
\newcommand{\Sol}{\textbf{Solution}.}

\section{Problem 1}
Briefly explain the following terms or concepts:

{\bf a.} What information can you learn when you analyze the statistical
information of the distances between the PUF responses from the
same PUF instance with the same challenge?

\newcommand{\p}{\mathcal{P}}
\newcommand{\puf}{\textsf{puf}}
\newcommand{\dist}{\textbf{dist}}
\Sol The following items can be learned:
\begin{itemize}
\item The distribution of a PUF response {\bf intra} distance:
$$D_{\puf_i}^{intra} := \dist[Y_i(x); Y'_i(x)],$$
with $Y_i(x)$ and $Y'_i(x)$ two distinct random evaluations of the same PUF instance $\puf_i$ on the same challenge $x$. $\dist[;]$ denotes Hamming distance between two random variables.
\item For the \emph{reproducibility} of a PUF class, distribution of $D_{\puf_i}^{intra}$ has some statistical properties.
\item Estimation of mean ($\mu_{\p}^{intra}$), standard deviation ($\sum[D_{\p}^{intra}]$), histogram and order statistics should be learned.
\end{itemize}

{\bf b.} What is the method to measure the inter-distance of the responses
for a PUF instance?

\Sol Quantitative tests given the same challenge on different PUFs.


{\bf c.} Suppose that you have two different PUFs. One of them is based
on SRAM PUF with $\mu_{\p}^{inter} = 49.59$ and $\sigma_{\p}^{inter} = 0.33$. The other one is based on Latch PUF with $\mu_{\p}^{inter} = 37.01$ and $\sigma_{\p}^{inter} = 1.23$. Which one is closer to ideal PUF by looking at the given statistical
information of the inter distances and why?

\Sol SRAM PUF is better than Latch PUF: $\mu_{\p}^{inter} > \mu_{\p}^{inter}$ says SRAM PUF has larger inter-distance, and $\sigma_{\p}^{inter} < \sigma_{\p}^{inter}$ says the inter-distance distribution of SRAM PUF is more steady.

{\bf d.} What is the reason of the noise in arbiter PUF?

\Sol 

{\bf e.} What is the way of making PUF responses uniformly random?

\Sol 

\section{Problem 2}
Suppose that you have a physical uncloneable function (PUF) with
intra-distance $\leq 8$ and inter-distance $52 \leq D_{\p}^{inter} \leq 73$. The PUF
output length is 127-bit. You are supposed to generate a 64-bit secret
key.

{\bf a.} Select $N$, $K$, and $t$ parameters for a BCH code to remove the noise.

\Sol 

\newcommand{\GF}{\textrm{GF}}
{\bf b.} What is the value of $m \in \GF(2^m)$ for this BCH code?

\Sol 

\section{Problem 3}

(Encoding BCH Codes) A BCH (15, 7, 2) code over $\GF(2^4)$ is given where $\alpha$ is a primitive polynomial element in the field and $p(X) = X^4 + X^3 + 1$.

{\bf a.} Find the generator polynomial for the code.

\Sol 

{\bf b.} Compute the codeword for the message sequence (0, 1, 1, 1, 0, 0, 0)
where its polynomial representation is $u(X) = X^3 + X^2 + X$.

\Sol 

\end{document}