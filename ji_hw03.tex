\documentclass[12pt]{article}
\usepackage{graphicx}
\usepackage{fullpage}
\usepackage{amsmath}
\usepackage{amsfonts}
\usepackage{amssymb}
\usepackage{amsthm}
\usepackage{algpseudocode}
\usepackage{algorithm}

\setlength{\parindent}{0cm} 

\newtheorem{proposition}{Proposition}
\newtheorem{claim}{Claim}
\theoremstyle{plain}

\begin{document}

\title{CS 290T Computational Algebra Homework Assignment 03}
\author{Shiyu Ji}
\date{\today}
\maketitle

\newcommand{\m}[1]{\begin{bmatrix}#1\end{bmatrix}}
\newcommand{\rank}[1]{\operatorname{rank}(#1)}
\newcommand{\F}{\mathbb{F}}

\section{Problem 1}
Briefly explain the following terms or concepts:

{\bf a.} What information can you learn when you analyze the statistical
information of the distances between the PUF responses from the
same PUF instance with the same challenge?

{\bf b.} What is the method to measure the inter-distance of the responses
for a PUF instance?

\newcommand{\p}{\mathcal{P}}
{\bf c.} Suppose that you have two different PUFs. One of them is based
on SRAM PUF with $\mu_{\p}^{inter} = 49.59$ and $\sigma_{\p}^{inter} = 0.33$. The other one is based on Latch PUF with $\mu_{\p}^{inter} = 37.01$ and $\sigma_{\p}^{inter} = 1.23$. Which one is closer to ideal PUF by looking at the given statistical
information of the inter distances and why?

{\bf d.} What is the reason of the noise in arbiter PUF?

{\bf e.} What is the way of making PUF responses uniformly random?

\section{Problem 2}
Suppose that you have a physical uncloneable function (PUF) with
intra-distance $\leq 8$ and inter-distance $52 \leq D_{\p}^{inter} \leq 73$. The PUF
output length is 127-bit. You are supposed to generate a 64-bit secret
key.

{\bf a.} Select $N$, $K$, and $t$ parameters for a BCH code to remove the noise.


\newcommand{\GF}{\textrm{GF}}
{\bf b.} What is the value of $m \in \GF(2^m)$ for this BCH code?


\section{Problem 3}

(Encoding BCH Codes) A BCH (15, 7, 2) code over $\GF(2^4)$ is given where $\alpha$ is a primitive polynomial element in the field and $p(X) = X^4 + X^3 + 1$.

{\bf a.} Find the generator polynomial for the code.

{\bf b.} Compute the codeword for the message sequence (0, 1, 1, 1, 0, 0, 0)
where its polynomial representation is $u(X) = X^3 + X^2 + X$.

\end{document}