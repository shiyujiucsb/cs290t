\documentclass[12pt]{article}
\usepackage{url,amsmath,setspace,amssymb,amsthm,amsfonts}
%\usepackage{hyperref}


\setlength{\oddsidemargin}{.25in}
\setlength{\evensidemargin}{.25in}
\setlength{\textwidth}{6.25in}
\setlength{\topmargin}{-0.4in}
\setlength{\textheight}{8.5in}

\setlength{\parindent}{0in}

\newcommand{\eqdef}{\stackrel{def}{=}}
\newcommand{\dom}{\mathcal{D}}
\newcommand{\ran}{\mathbb{R}}
\newcommand{\N}{\mathbb{N}}
\newcommand{\R}{\mathbb{R}}
\newcommand{\Z}{\mathbb{Z}}
\newcommand{\F}{\mathbb{F}}
\newcommand{\bits}{\{0,1\}}
\newcommand{\inr}{\in_{\mbox{\tiny R}}}
%\newcommand{\getsr}{\gets_{\mbox{\tiny R}}}
\newcommand{\getsr}{\stackrel{\$}{\gets}}
\newcommand{\st}{\mbox{ s.t. }}
\newcommand{\etal}{{\it et al }}
\newcommand{\into}{\rightarrow}

\newcommand{\Ex}{\mathbb{E}}
\newcommand{\e}{\epsilon}
\newcommand{\ee}{\varepsilon}
\newcommand{\ceil}[1]{{\lceil{#1}\rceil}}
\newcommand{\floor}[1]{{\lfloor{#1}\rfloor}}
\newcommand{\angles}[1]{\langle #1 \rangle}
\newcommand{\Com}{{\sf Com}}
\newcommand{\desc}{{\sf desc}}

\newcommand{\rightstep}[1]{%
$\underrightarrow{\quad #1 \quad}$ }

\newcommand{\leftstep}[1]{%
$\underleftarrow{\quad #1 \quad}$ }

\newcommand{\Adv}{\mathsf{Adv}}
\newcommand{\poly}{\mathsf{poly}}

\newcommand{\SVP}{\textsf{SVP}}
\newcommand{\CVP}{\textsf{CVP}}

\newcommand{\tab}{\hspace{0.3in}}

\newcommand{\io}{$i\mathcal{O}$}
\newcommand{\dio}{$di\mathcal{O}$}
%%%%%%%%%%%%%%%%%%%%%%%%%%%%
% Theorems & Definitions


\newtheorem{theorem}{Theorem}[section]

\newtheorem{claim}[theorem]{Claim}
\newtheorem{subclaim}{Claim}[theorem]
\newtheorem{proposition}[theorem]{Proposition}
\newtheorem{lemma}[theorem]{Lemma}
\newtheorem{corollary}[theorem]{Corollary}
\newtheorem{conjecture}[theorem]{Conjecture}

\theoremstyle{definition}
\newtheorem{definition}[theorem]{Definition}
\newtheorem{construction}[theorem]{Construction}
\newtheorem{example}[theorem]{Example}
\newtheorem{algorithm1}[theorem]{Algorithm}
\newtheorem{protocol}[theorem]{Protocol}
\newtheorem{remark}[theorem]{Remark}
\newtheorem{observation}[theorem]{Observation}
\newtheorem{assumption}[theorem]{Assumption}
\newtheorem{fact}[theorem]{Fact}

%\bibliographystyle{plain}
\usepackage{tikz}
\usetikzlibrary{calc,decorations.pathreplacing}

\begin{document}
%\handout{}{\today{}}{}
\title{On Lattice-based Physically Unclonable Functions}
\author{Shiyu Ji}
\date{\today}
\maketitle

{\it Abstract}. 
In this short paper we propose a lattice-based construction on physically unclonable functions (PUFs). We will also analyze its security by considering the existing notions summarized by Sadeghi et al. \cite{sadeghi2016towards}. By long-standing hardness assumptions on lattice problems (e.g. \SVP), it turns out our construction can achieve most important security requirements with acceptable parameters. The construction itself is simple and efficient: there is no more operation except modular multiplications and additions. Thus it could lead to promising practical value. We also extend our construction to LWE-based case, and show that our extended LWE-based PUFs can achieve nontrivial min-entropy, which is a popular notion to assert the uncertainty in PUF output (as claimed in \cite{sadeghi2016towards}). 

\section{Introduction}


As an important algebraic structure, lattice has been studied for many years \cite{regev2009lattices}. However, there are still some long-standing hard problems on lattice computation, e.g., \emph{short vector problem} (\SVP), \emph{close vector problem} (\CVP), etc. These hardness assumptions give rise to many important cryptographic constructions. For instance, Goldreich et al. \cite{goldreich2011collision} give a collision resistant hash function (CRHF) scheme based on \SVP\, (Problem {\bf A2} in their paper). The idea of this CRHF gives rise to our PUF construction in this paper.

\bibliographystyle{alpha}
\bibliography{puf}
	
\end{document}
