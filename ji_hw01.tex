\documentclass[12pt]{article}
\usepackage{graphicx}
\usepackage{fullpage}
\usepackage{amsmath}
\usepackage{amsfonts}
\usepackage{amssymb}
\usepackage{amsthm}
\usepackage{algpseudocode}
\usepackage{algorithm}

\setlength{\parindent}{0cm} 

\newtheorem{proposition}{Proposition}
\theoremstyle{plain}

\begin{document}

\title{CS 290T Computational Algebra Homework Assignment 01}
\author{Shiyu Ji}
\date{\today}
\maketitle

\newcommand{\m}[1]{\begin{pmatrix}#1\end{pmatrix}}
\newcommand{\rank}[1]{\operatorname{rank}(#1)}

Consider the set of linear equations with the Hilbert matrix of dimension 3
$$\m{1 & \frac{1}{2} & \frac{1}{3} \\
\frac{1}{2} & \frac{1}{3} & \frac{1}{4} \\
\frac{1}{3} & \frac{1}{4} & \frac{1}{5} \\}
\m{x_1 \\ x_2 \\ x_3}=
\m{\frac{1}{4} \\ \frac{1}{5} \\ \frac{1}{6}}$$
whose entries can be scaled up by $\mathrm{lcm}(2,3,4,5,6) = 60$ to obtain the integer system of equations
$$\m{60 & 30 & 20 \\
30 & 20 & 15 \\
20 & 15 & 12}
\m{x_1 \\ x_2 \\ x_3}=
\m{15 \\ 12 \\ 10}.$$
The determinant and the solution can be found using Mathematica as
$$d=100 \quad \mathrm{and}\quad 
\m{x_1 \\ x_2 \\ x_3}=
\m{\frac{1}{20} \\ -\frac{3}{5} \\ \frac{3}{2}}.$$
(1) Solve this integer system of equations using the multiple-modulus congruence technique with the moduli set $(n_1, n_2, n_3, n_4) = (7, 8, 9, 11)$. Add one more modulus if necessary. Use the MRC algorithm at the
last step.

{\bf Solution}:

\emph{Step 1}. Let $(n_1, n_2, n_3, n_4) = (7, 8, 9, 11)$ be the set of pairwise relative prime numbers. Hence we need to solve 4 linear systems $Ax_i = b \mod n_i$ as follows. All the computations can be done by Gaussian elimination. The details are not presented since they are routine and tedious.
$$\m{4 & 2 & 6 \\
2 & 6 & 1 \\
6 & 1 & 5}
\m{x_1 \\ x_2 \\ x_3}=
\m{1 \\ 5 \\ 3} \mod 7$$
gives $x_1 = \m{x_1 & x_2 & x_3}_1^T = \m{6 & 5 & 5}^T$.
$$\m{4 & 6 & 4 \\
6 & 4 & 7 \\
4 & 7 & 4}
\m{x_1 \\ x_2 \\ x_3}=
\m{7 \\ 4 \\ 2} \mod 8$$
gives $x_2 = \m{x_1 & x_2 & x_3}_2^T = \m{6 & 5 & 5}^T$.

(2) Solve this integer system of equations using Dixon’s algorithm for $p = 13$ and as large $k$ as needed. Determine the minimum $k$ value which
finds the solution.

{\bf Solution}:

\end{document}
