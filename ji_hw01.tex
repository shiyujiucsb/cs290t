\documentclass[12pt]{article}
\usepackage{graphicx}
\usepackage{fullpage}
\usepackage{amsmath}
\usepackage{amsfonts}
\usepackage{amssymb}
\usepackage{amsthm}
\usepackage{algpseudocode}
\usepackage{algorithm}

\setlength{\parindent}{0cm} 

\newtheorem{proposition}{Proposition}
\theoremstyle{plain}

\begin{document}

\title{CS 290T Computational Algebra Homework Assignment 01}
\author{Shiyu Ji}
\date{\today}
\maketitle

\newcommand{\m}[1]{\begin{pmatrix}#1\end{pmatrix}}
\newcommand{\rank}[1]{\operatorname{rank}(#1)}

Consider the set of linear equations with the Hilbert matrix of dimension 3
$$\m{1 & \frac{1}{2} & \frac{1}{3} \\
\frac{1}{2} & \frac{1}{3} & \frac{1}{4} \\
\frac{1}{3} & \frac{1}{4} & \frac{1}{5} \\}
\m{x_1 \\ x_2 \\ x_3}=
\m{\frac{1}{4} \\ \frac{1}{5} \\ \frac{1}{6}}$$
whose entries can be scaled up by $\mathrm{lcm}(2,3,4,5,6) = 60$ to obtain the integer system of equations
$$\m{60 & 30 & 20 \\
30 & 20 & 15 \\
20 & 15 & 12}
\m{x_1 \\ x_2 \\ x_3}=
\m{15 \\ 12 \\ 10}.$$
The determinant and the solution can be found using Mathematica as
$$d=100 \quad \mathrm{and}\quad 
\m{x_1 \\ x_2 \\ x_3}=
\m{\frac{1}{20} \\ -\frac{3}{5} \\ \frac{3}{2}}.$$

\section{Problem 1}
Solve this integer system of equations using the multiple-modulus congruence technique with the moduli set $(n_1, n_2, n_3, n_4) = (7, 9, 11, 13)$. Add one more modulus if necessary. Use the MRC algorithm at the
last step.

{\bf Solution}:

\emph{Step 1}. Let $(n_1, n_2, n_3, n_4) = (7, 9, 11, 13)$ be the set of pairwise relative prime numbers. Hence we need to solve 4 linear systems $Ax_i = b \mod n_i$ as follows. All the computations can be done by Gaussian elimination. The details are not presented since they are routine and tedious.
$$\m{4 & 2 & 6 \\
2 & 6 & 1 \\
6 & 1 & 5}
\m{x_{11} \\ x_{12} \\ x_{13}}=
\m{1 \\ 5 \\ 3} \mod 7$$
gives $x_1 = \m{x_{11} & x_{12} & x_{13}}_1^T = \m{6 & 5 & 5}^T$.
$$\m{6 & 3 & 2 \\
3 & 2 & 6 \\
2 & 6 & 3}
\m{x_{21} \\ x_{22} \\ x_{23}}=
\m{6 \\ 3 \\ 1} \mod 9$$
gives $x_2 = \m{x_{21} & x_{22} & x_{23}}_2^T = \m{5 & 3 & 6}^T$.
$$\m{5 & 8 & 9 \\
8 & 9 & 4 \\
9 & 4 & 1}
\m{x_{31} \\ x_{32} \\ x_{33}}=
\m{4 \\ 1 \\ 10} \mod 11$$
gives $x_3 = \m{x_{31} & x_{32} & x_{33}}_2^T = \m{5 & 6 & 7}^T$.
$$\m{8 & 4 & 7 \\
4 & 7 & 2 \\
7 & 2 & 12}
\m{x_{41} \\ x_{42} \\ x_{43}}=
\m{2 \\ 12 \\ 10} \mod 13$$
gives $x_4 = \m{x_{41} & x_{42} & x_{43}}_2^T = \m{2 & 2 & 8}^T$.

Hence we have found four solutions of $x_i$'s.

\emph{Step 2}. Compute the determinant $d_i = d \mod n_i$ for each $i$. That is, 
$$d_1 = det(A \mod 7) \mod 7 = -96 \mod 7 = 2,$$
$$d_2 = det(A \mod 9) \mod 9 = -143 \mod 9 = 1,$$ 
$$d_3 = det(A \mod 11) \mod 11 = -252 \mod 11 = 1,$$
$$d_4 = det(A \mod 13) \mod 13 = 217 \mod 13 = 9.$$

\emph{Step 3}. Compute $y_i = d_i x_i \mod n_i$ for each $i$.
$$y_1 = d_1 x_1 \mod n_1 = 2 \cdot \m{6 & 5 & 5}^T \mod 7 = \m{5 & 3 & 3},$$ 
$$y_2 = d_2 x_2 \mod n_2 = 1 \cdot \m{5 & 3 & 6}^T \mod 9 = \m{5 & 3 & 6},$$ 
$$y_3 = d_3 x_3 \mod n_3 = 1 \cdot \m{5 & 6 & 7}^T \mod 11 = \m{5 & 6 & 7},$$ 
$$y_4 = d_4 x_4 \mod n_4 = 9 \cdot \m{2 & 2 & 8}^T \mod 13 = \m{5 & 5 & 7}.$$ 

\emph{Step 4}. Apply MRC algorithm on $\{y_1,y_2,y_3,y_4\}$ to compute $y$.
We first compute all the $c_{ij}$ for $1\leq i < j \leq 4$.
$$c_{12} = n_1^{-1} \mod n_2 = 7^{-1} \mod 9 = 4.$$
$$c_{13} = n_1^{-1} \mod n_3 = 7^{-1} \mod 11 = 8.$$
$$c_{14} = n_1^{-1} \mod n_4 = 7^{-1} \mod 13 = 2.$$
$$c_{23} = n_2^{-1} \mod n_3 = 9^{-1} \mod 11 = 5.$$
$$c_{24} = n_2^{-1} \mod n_4 = 9^{-1} \mod 13 = 3.$$
$$c_{34} = n_3^{-1} \mod n_4 = 11^{-1} \mod 13 = 6.$$
To compute $y$, we compute all the $s_{ij}$ for $1\leq i\leq j \leq 4$. The iterative equation is $s_{i+1,j+1} = (s_{i+1,j} - s_{j,j})\cdot c_{j,i+1} \mod n_{i+1}$. 
$$s_{11} = y_1 = \m{5 & 3 & 3}^T.$$
$$s_{21} = y_2 = \m{5 & 3 & 6}^T,$$
$$s_{22} = (s_{21}-s_{11})\cdot c_{12} \mod n_2 = \m{0 & 0 & 3}^T\times 4 \mod 9 = \m{0 & 0 & 3}^T.$$
$$s_{31} = y_3 = \m{5 & 6 & 7}^T,$$
$$s_{32} = (s_{31}-s_{11})\cdot c_{13} \mod n_3 = \m{0 & 3 & 4}^T\times 8 \mod 11 = \m{0 & 2 & 10}^T,$$
$$s_{33} = (s_{32}-s_{22})\cdot c_{23} \mod n_3 = \m{0 & 2 & 7}^T\times 5 \mod 11 = \m{0 & 10 & 2}^T.$$
$$s_{41} = y_4 = \m{5 & 5 & 7}^T,$$
$$s_{42} = (s_{41}-s_{11})\cdot c_{14} \mod n_4 = \m{0 & 2 & 4}^T\times 2 \mod 13 = \m{0 & 4 & 8}^T,$$
$$s_{43} = (s_{42}-s_{22})\cdot c_{24} \mod n_4 = \m{0 & 4 & 5}^T\times 3 \mod 13 = \m{0 & 12 & 2}^T,$$
$$s_{44} = (s_{43}-s_{33})\cdot c_{34} \mod n_4 = \m{0 & 2 & 0}^T\times 6\mod 13 = \m{0 & 12 & 0}^T.$$
We have $y = s_{11} + s_{22}n_1 + s_{33}n_1n_2 + s_{44}n_1n_2n_3 = \m{5 \\ 8949 \\ 150}$.
Note that 8949 is more than the half of $N=9009$. Hence we adjust it to be $8949-9009=-60$. The final is $y=\m{5 \\ -60 \\ 150}$.

\emph{Step 5}. Apply MRC algorithm on $\{d_1,d_2,d_3,d_4\}$ to compute $d$. 
All the $c_{ij}$'s have been computed in Step 4. 
We next compute all the $s_{ij}$ for $1\leq i\leq j \leq 4$. The iterative equation is $s_{i+1,j+1} = (s_{i+1,j} - s_{j,j})\cdot c_{j,i+1} \mod n_{i+1}$. 
$$s_{11} = d_1 = 2.$$
$$s_{21} = d_2 = 1,$$
$$s_{22} = (s_{21}-s_{11})\cdot c_{12} \mod n_2 = (1-2)\times 4 \mod 9 = 5.$$
$$s_{31} = d_3 = 1,$$
$$s_{32} = (s_{31}-s_{11})\cdot c_{13} \mod n_3 = (1-2)\times 8 \mod 11 = 3,$$
$$s_{33} = (s_{32}-s_{22})\cdot c_{23} \mod n_3 = (3-5)\times 5 \mod 11 = 1.$$
$$s_{41} = d_4 = 9,$$
$$s_{42} = (s_{41}-s_{11})\cdot c_{14} \mod n_4 = (9-2)\times 2 \mod 13 = 1,$$
$$s_{43} = (s_{42}-s_{22})\cdot c_{24} \mod n_4 = (1-5)\times 3 \mod 13 = 1,$$
$$s_{44} = (s_{43}-s_{33})\cdot c_{34} \mod n_4 = (1-1)\times 6\mod 13 = 0.$$
We have $d = s_{11} + s_{22}n_1 + s_{33}n_1n_2 + s_{44}n_1n_2n_3 = 2+5\times 7+1\times 7\times 9 + 0 = 100$.
We can also use the traditional CRT mapping to solve $d$. See Appendix.

\emph{Step 6}. The exact rational solution is given as
$$x=\frac{1}{d}y = \frac{1}{100} \m{5 \\ -60 \\ 150} = \m{\frac{1}{20} \\ -\frac{3}{5} \\ \frac{3}{2}}.$$

\section{Problem 2}
Solve this integer system of equations using Dixon’s algorithm for $p = 13$ and as large $k$ as needed. Determine the minimum $k$ value which
finds the solution.

{\bf Solution}:

\emph{Step 1}. Compute the inverse of the matrix $A \mod p$. This can be done by using GEA. 
$$C = A^{-1}\mod 13 = \m{6 & 2 & 7 \\ 2 & 11 & 10 \\ 7 & 10 & 3}.$$

\emph{Step 2}. We start with the initial $b$ for the iteration.
$$b_0 = b = \m{15 & 12 & 10}^T.$$
$$x_0 = Cb_0 \mod p = \m{6 & 2 & 7 \\ 2 & 11 & 10 \\ 7 & 10 & 3}\m{15 \\ 12 \\ 10}\mod 13 = \m{2\\2\\8}.$$
$$b_1 = \frac{1}{p}(b_0-Ax_0)= \frac{1}{13}\left(\m{15 \\ 12 \\ 10}-\m{60 & 30 & 20 \\
30 & 20 & 15 \\
20 & 15 & 12}\m{2\\2\\8}\right) = \m{-25\\-16\\-12}$$
$$x_1 = Cb_1 \mod p = \m{6 & 2 & 7 \\ 2 & 11 & 10 \\ 7 & 10 & 3}\m{-25 \\ -16 \\ -12}\mod 13 = \m{7\\5\\6}.$$
$$b_2 = \frac{1}{p}(b_1-Ax_1)= \frac{1}{13}\left(\m{-25\\-16\\-12}-\m{60 & 30 & 20 \\
30 & 20 & 15 \\
20 & 15 & 12}\m{7\\5\\6}\right) = \m{-55\\-34\\-25}.$$
$$x_2 = Cb_2 \mod p = \m{6 & 2 & 7 \\ 2 & 11 & 10 \\ 7 & 10 & 3}\m{-55\\-34\\-25} \mod 13 = \m{12\\7\\6}.$$
$$b_3 = \frac{1}{p}(b_2-Ax_2)= \m{-85\\-48\\-34}.$$
$$x_3 = Cb_3 \mod p = \m{1\\2\\6}.$$
$$b_4 = \frac{1}{p}(b_3-Ax_3) = \m{-25\\-16\\-12}.$$
We first {\bf try} $k=2$.
$$\overline{x} = x_0 + x_1 p = \m{2\\2\\8} + \m{7\\5\\6} \cdot 13 = \m{93\\67\\86}.$$

\emph{Step 3}. This is Euclidean step. Start with $u_{-1} = p^2 = 169$, $u_0=\overline{x} = \m{93 & 67 & 86}^T$, $v_{-1} = 0$ and $v_0=1$. Let $q_0 = \lfloor \frac{u_{-1}}{u_0} \rfloor = \m{1 & 2 & 1}^T$. Hence the halting threshold is $h = \sqrt{p^k} = 13$. Compute
$$u_1 = u_{-1} - q_0\cdot u_0 = 169 - \m{1 & 2 & 1}^T \cdot \m{93 & 67 & 86}^T = \m{76 & 35 & 83}^T.$$
$$v_1 = v_{-1} + q_0\cdot v_0 = \m{1 & 2 & 1}^T \cdot 1 = \m{1 & 2 & 1}^T.$$
Note that all $u_1$'s are larger than $13$. Hence
$$q_1 = \lfloor \frac{u_0}{u_1} \rfloor = \m{1 & 1 & 1}^T.$$
$$u_2 = u_0 - q_1\cdot u_1 = \m{93 & 67 & 86}^T - \m{1 & 1 & 1}^T \cdot \m{76 & 35 & 83}^T = \m{17 & 32 & 3}^T.$$
$$v_2 = v_0 + q_1\cdot v_1 = 1 + \m{1 & 1 & 1}^T \cdot \m{1 & 2 & 1}^T = \m{2&3&2}^T.$$
The third entry has satisfied the terminating condition $u_{2,3} = 3 < 13$. Thus its final output is 
$$(-1)^2u_{2,3}/v_{2,3} = 3/2.$$
Note that 17 and 32 are greater than $13$. Hence we continue. Let $u_1 = \m{76 & 35}^T$ and $u_2 = \m{17 & 32}^T$.
$$q_2 = \lfloor \frac{u_1}{u_2} \rfloor = \m{4 & 1}^T.$$
$$u_3 = u_1 - q_2\cdot u_2 = \m{76 & 35}^T - \m{4 & 1}^T \cdot \m{17 & 32}^T = \m{8 & 3}^T.$$
$$v_3 = v_1 + q_2\cdot v_2 = \m{1 & 2}^T + \m{4 & 1}^T \cdot \m{2&3}^T = \m{9 & 5}^T.$$
The second entry is less than 13. So it can terminate. The final output is 
$$(-1)^3u_{2,2}/v_{2,2} = -3/5.$$
Note that even though the first entry 8 is also less than $13$, the final output 
$$(-1)^3u_{2,1}/v_{2,1} = -8/9$$
 is {\bf incorrect}!
But interestingly, if we do one more loop for the first entry, we can obtain the correct solution!
$$q_3 = \lfloor \frac{17}{8} \rfloor = 2.$$
$$u_4 = u_2 - q_3\cdot u_3 = 17 - 2\times 8 = 1.$$
$$v_4 = v_2 + q_3\cdot v_3 = 2 + 2\times 9 = 20.$$
Finally, we have $i=4$, and now the final output is $1/20$, which is correct.
However this {\it violates} Dixon's algorithm! This implies $k=2$ is too small, and gives us the hint: increasing $k$ by 1 could give the correct solution.

\emph{Step 4}. By $x_i = (-1)^i u_i/v_i$ and $i=3$, we have the final solution
$$x = \m{-\frac{8}{9} & -\frac{3}{5} & \frac{3}{2}}^T,$$
which is {\bf incorrect}!

Note that it is \emph{impossible} to correctly calculate the first entry in this way with $k=2$. I have written a Matlab program to test the first 8 $k$'s. See Table \ref{tab:1} for the results. See Appendix for the code.
\begin{table}
\centering
\caption{Euclidean step with $s=93$.}
\begin{tabular}{c || c| c| c| c| c| c |c |c}
\hline
$k$ & 1 & 2 & 3 & 4 & 5 & 6 & 7 & 8\\
\hline
$x$ & 2 & -8/9 & 35/24 & -10/307 & -37/3992 & -16/51901 & -7/214682 & -1/1253043 \\
\hline
\end{tabular}
\label{tab:1}
\end{table}
As $k$ grows, the final output $x$ approaches to 0, and is always negative. This is because when $k$ is very large (e.g. $k>4$), 
$$u_1 < 93 < 169 = \sqrt{13^4} < \sqrt{p^k}.$$
Hence the algorithm always terminates at the first round: $i=1$. As a result the final output should be negative. However the correct answer $1/20$ is positive. Thus it is impossible to correctly solve the problem by assuming $k=2$.

Therefore, we {\bf try} $k=3$.
$$\overline{x} = x_0 + x_1p+x_2p^2 = \m{2\\2\\8} + \m{7\\5\\6} \cdot 13 + \m{4\\10\\6} \cdot 13^2 = \m{769\\1757\\1100}.$$

\emph{Step 3'}. This is Euclidean step. Start with $u_{-1} = p^3 = 2197$, $u_0=\overline{x} = \m{769& 1757& 1100}^T$, $v_{-1} = 0$ and $v_0=1$. Let $q_0 = \lfloor \frac{u_{-1}}{u_0} \rfloor = \m{2 & 1 & 1}^T$. Hence the halting threshold is $h = \sqrt{p^k} = 13^{3/2}$. Compute
$$u_1 = u_{-1} - q_0\cdot u_0 = 2197 - \m{2 & 1 & 1}^T \cdot \m{769 & 1757 & 1100}^T = \m{659 & 440 & 1097}^T.$$
$$v_1 = v_{-1} + q_0\cdot v_0 = \m{2 & 1 & 1}^T \cdot 1 = \m{2 & 1 & 1}^T.$$
Note that all $u_1$'s are larger than $13^{3/2}$. Hence
$$q_1 = \lfloor \frac{u_0}{u_1} \rfloor = \m{1 & 3 & 1}^T.$$
$$u_2 = u_0 - q_1\cdot u_1 = \m{769&1757&1100}^T - \m{1 & 3 & 1}^T \cdot \m{659 & 440 & 1097}^T = \m{110 & 437 & 3}^T.$$
$$v_2 = v_0 + q_1\cdot v_1 = 1 + \m{1 & 3 & 1}^T \cdot \m{2 & 1 & 1}^T = \m{3&4&2}^T.$$
The third entry has satisfied the terminating condition $u_{2,3} = 3 < 13$. Thus its final output is 
$$(-1)^2u_{2,3}/v_{2,3} = 3/2.$$
Note that 110 and 437 are greater than $13^{3/2}$. Hence we continue. Let $u_1 = \m{659 & 440}^T$ and $u_2 = \m{110 & 437}^T$.
$$q_2 = \lfloor \frac{u_1}{u_2} \rfloor = \m{5 & 1}^T.$$
$$u_3 = u_1 - q_2\cdot u_2 = \m{659 & 440}^T - \m{5 & 1}^T \cdot \m{110 & 437}^T = \m{109 & 3}^T.$$
$$v_3 = v_1 + q_2\cdot v_2 = \m{2 & 1}^T + \m{5 & 1}^T \cdot \m{3&4}^T = \m{17 & 5}^T.$$
The second entry is less than $13^{3/2}$. So it can terminate. The final output is 
$$(-1)^3u_{2,2}/v_{2,2} = -3/5.$$
Note that even though the first entry 109 is still larger than $13^{3/2}$. Hence we continue. 
$$q_3 = \lfloor \frac{u_2}{u_3} \rfloor = 1.$$
$$u_4 = u_2 - q_3\cdot u_3 = 110 - 109 = 1.$$
$$v_4 = v_2 + q_3\cdot v_3 = 3 + 17 = 20.$$
Now even the first entry can terminate. Thus the final answer of the first entry is $1/20$, since $i=4$ is even.

\emph{Step 4'}. Putting things together we get the final solution:
$$x = \m{\frac{1}{20} & -\frac{3}{5} & \frac{3}{2}}^T,$$
which is correct!

Hence our conclusion is: $k=3$ is the {\bf minimum} number that can give us correct solution.

\section{Appendix}
\subsection{Traditional CRT Method}
We can solve $d$ using traditional CRT as well. 
Let $N = \prod_{k=1}^4 n_i = 9009$.
$$\begin{aligned}
d =& \sum_{k=1}^4 d_i \left(\left(\frac{N}{n_i}\right)^{-1} \mod n_i \right) \frac{N}{n_i}  (\mod N)\\
=& 2\times(1287^{-1} \mod 7)\times 1287 + 1\times(1001^{-1} \mod 9)\times 1001 \\
&+ 1\times(819^{-1} \mod 11)\times 819 + 9\times(693^{-1} \mod 13)\times 693 (\mod 9009)\\
=& 2\times 6\times 1287 + 1\times 5\times 1001 + 1\times 9\times 819 + 9\times 10\times 693 (\mod 9009)\\
=& 90190 \mod 9009 = 100.
\end{aligned}$$

\subsection{Matlab Code for Dixon's Method}
\begin{verbatim}
function y = Dixon(k)
  C = [6,2,7;2,11,10;7,10,3];
  A = [60,30,20;30,20,15;20,15,12];
  p = 13;
  b = [15;12;10];
  i = 0;
  xbar = [0;0;0];
  while i<k
    x = mod(C*b,p);
    b = (b-A*x)./p;
    xbar = xbar + (x.*(p.^i));
    i = i+1;
  end
  y = [0;0;0];
  y(1) = Euclidean_Step(xbar(1), k);
  y(2) = Euclidean_Step(xbar(2), k);
  y(3) = Euclidean_Step(xbar(3), k);
end
\end{verbatim}

\subsection{Matlab Code for Euclidean Step}
\begin{verbatim}
function y = Euclidean_Step(s,k)
  p = 13;
  h = (p.^k).^0.5;

  u_neg_1 = 13.^k;
  u0 = s;
  v_neg_1 = 0;
  v0 = 1;

  i=0;
  u1 = h+1;
  while u1>=h 
    q0 = floor(u_neg_1./u0);
    u1 = u_neg_1 - q0*u0;
    v1 = v_neg_1 + q0*v0;
    if u1>h 
      u_neg_1 = u0;
      v_neg_1 = v0;
      u0 = u1;
      v0 = v1;
    end 
    i = i+1;
  end
  y = (-1).^i*u1./v1;
end
\end{verbatim}

\end{document}
